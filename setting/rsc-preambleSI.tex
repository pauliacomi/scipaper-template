% !TEX root = ../manuscript-SI.tex

%%%%%%%%%%%%%%%%%%%%%% TEMPLATE
% RCS Template:
\usepackage{times,mathptmx}
\usepackage{sectsty}
\usepackage{fnpos}
\usepackage{droidsans}
\usepackage{charter}
\usepackage{setspace}
\usepackage[compact]{titlesec}

%%%%%%%%%%%%%%%%%%%%%% 
%%%%%%%%%%%%%%%%%%%%%% OWN
%%%%%%%%%%%%%%%%%%%%%% Geometry and other page improvements

\usepackage[
  left=2.0cm, 
  right=2.0cm, 
  top=2.0cm, 
  bottom=2.0cm
]{geometry}

\def\thepage{S-\arabic{page}}
\renewcommand*\thefigure{S\arabic{figure}}
\renewcommand*\thetable{S\arabic{table}}
\AtBeginDocument{%
    \renewcommand*\citenumfont[1]{S#1}%
    \renewcommand*\bibnumfmt[1]{(S#1)}%
}

%%%%%%%%%%%%%%%%%%%%%% Encodings, fonts and colours

% Font
% \usepackage{fourier}

% Font encodings
\usepackage[TS1,T1]{fontenc}

% UNICODE recognition:
\usepackage[utf8]{inputenc}

% For colour definitions
\usepackage{xcolor}

%%%%%%%%%%%%%%%%%%%%%% Bibliography packages and settings

% Natbib package for text references
\usepackage[
    % sectionbib,
    % round,
    super,
    comma,
    sort&compress
]{natbib}

% Moves references after punctuation
% \usepackage{natmove}
% \renewcommand*{\natmovechars}{.}

% Adds bibliography to TOC
% \usepackage[nottoc]{tocbibind}

%%%%%%%%%%%%%%%%%%%%%% Science-related packages

% AMS maths packages
\usepackage{
    amsmath,
    amsfonts,
    amsthm,
    % amssymb
}

% In-line fractions
\usepackage{xfrac}

% The SIunitx package enables the \SI{}{} command.
\usepackage{siunitx}

% The mchem package for formula subscripts using \ce{}
\usepackage[version=3]{mhchem}

%%%%%%%%%%%%%%%%%%%%%% Packages for graphs, tables and listings

% Graphics package and path to graphics.
\usepackage{graphicx}
\graphicspath{ {figs/} }

% Writing over figures
% \usepackage[percent]{overpic} 

% Wrapping around figures
% \usepackage{wrapfig}

% For better looking captions.
\usepackage{caption}
\captionsetup{
    margin=10pt,
    font=footnotesize,
    labelfont=bf,
    format=plain
}

% For the subfigures
\usepackage{subcaption}

% For float barriers and such
\usepackage{float}
\usepackage[section]{placeins}

% Better tables 
\usepackage{
    booktabs,
    makecell,
    array,
    multirow,
    tabularx
}

% For code snippets
\usepackage{listings}

%%%%%%%%%%%%%%%%%%%%%% References and bookmarks

% Xr:
% To reference another document, in this case the SI
\usepackage{xr, xr-hyper}

% Hyperref:
% This package makes all references within your document clickable. By default, these references will become boxed and colored. This is turned back to normal with the \hypersetup command below.
\usepackage{hyperref}

% Adding package bookmark improves bookmarks handling.
% More features and faster updated bookmarks.
% \usepackage{bookmark}

\hypersetup{
    % bookmarks=true,                         %% Acrobat bookmarks (default)
    % bookmarksnumbered=true,                 %% Add numbers to bookmarks
    % bookmarksopen=true,                     %% bookmarks default open
    % bookmarksopenlevel=2,                   %% Level 2 bookmarks
    % backref=true,                           %% add links in bib (default)
    % pagebackref=true,                       %% in the bibliography (default)
    % hyperindex=true,                        %% in the index (default)
    % breaklinks=true,                        %% break lines if long
    % colorlinks=false,                       %% colour links
    % urlcolor=blue,                          %% link colour
    % citecolor=blue,	                        %% bibliography link color
    % linkcolor=blue,	                        %% internal link color
    % anchorcolor=blue,                       %% anchor link color
    % linktocpage=false,                      %% pagenumber link in TOC
    pdfborder={0 0 0}                       %% pdf border settings
}

%%%%%%%%%%%%%%%%%%%%%% Miscellaneous

% Used for notes and annotations.
\usepackage{todonotes}