% !TEX root = ../manuscript.tex

%%%%%%%%%%%%%%%%%%%%%% TEMPLATE
% RCS Template:
\usepackage{extsizes}
\usepackage[super,sort&compress,comma]{natbib}
\usepackage[version=3]{mhchem}
\usepackage[left=1.5cm, right=1.5cm, top=1.785cm, bottom=2.0cm]{geometry}
\usepackage{balance}
\usepackage{times,mathptmx}
\usepackage{sectsty}
\usepackage{lastpage}
\usepackage[
  format=plain,
  justification=justified,
  singlelinecheck=false,
  font={stretch=1.125,small,sf},
  labelfont=bf,
  labelsep=space
]{caption}
\usepackage{float}
\usepackage{fancyhdr}
\usepackage{fnpos}
\usepackage[english]{babel}
\addto{\captionsenglish}{%
  \renewcommand{\refname}{Notes and references}
}
\usepackage{array}
\usepackage{droidsans}
\usepackage{charter}
\usepackage[T1]{fontenc}
\usepackage[usenames,dvipsnames]{xcolor}
\usepackage{setspace}
\usepackage[compact]{titlesec}

\definecolor{cream}{RGB}{222,217,201}

%%%%%%%%%%%%%%%%%%%%%% 
%%%%%%%%%%%%%%%%%%%%%% OWN
%%%%%%%%%%%%%%%%%%%%%% Encodings, fonts and colours

% Font encodings
% \usepackage{fourier}
% \usepackage[TS1,T1]{fontenc}

% UNICODE recognition:
% This package should recognise any UNICODE characters in the text and automatically replace them with their standard macros
\usepackage[utf8]{inputenc}

% For colour definitions
% \usepackage{xcolor}

%%%%%%%%%%%%%%%%%%%%%% Bibliography packages and settings

% Natbib package for text references
% \usepackage[
%     sectionbib,
%     round,
%     super,
%     comma,
%     sort&compress
% ]{natbib}

% Moves references after punctuation
% \usepackage{natmove}
% \renewcommand*{\natmovechars}{.}

% Adds bibliography to TOC
% \usepackage[nottoc]{tocbibind}

%%%%%%%%%%%%%%%%%%%%%% Science-related packages

% AMS maths packages
\usepackage{
    amsmath,
    amsfonts,
    amsthm,
    % amssymb
}

% In-line fractions
\usepackage{xfrac}

% The SIunitx package enables the \SI{}{} command.
\usepackage{siunitx}

% The mchem package for formula subscripts using \ce{}
% \usepackage[version=3]{mhchem} 

%%%%%%%%%%%%%%%%%%%%%% Packages for graphs, tables and listings

% Graphics package and path to graphics.
\usepackage{graphicx}
\graphicspath{ {figs/} }

% Writing over figures
% \usepackage[percent]{overpic} 

% Wrapping around figures
% \usepackage{wrapfig}

% For better looking captions.
% \usepackage{caption}
% \captionsetup{
%     margin=10pt,
%     font=footnotesize,
%     labelfont=bf,
%     format=plain
% }

% For the subfigures
\usepackage{subcaption}

% For float barriers and such
% \usepackage{float}
\usepackage[section]{placeins}

% Better tables 
\usepackage{
    booktabs,
    makecell,
    % array,
    multirow,
    tabularx
}

% For code snippets
\usepackage{listings}

%%%%%%%%%%%%%%%%%%%%%% References and bookmarks

% Xr:
% To reference another document, in this case the SI
\usepackage{xr, xr-hyper}

% Hyperref:
% This package makes all references within your document clickable. By default, these references will become boxed and colored. This is turned back to normal with the \hypersetup command below.
\usepackage{hyperref}

% Adding package bookmark improves bookmarks handling.
% More features and faster updated bookmarks.
% \usepackage{bookmark}

\hypersetup{
    % bookmarks=true,                         %% Acrobat bookmarks (default)
    bookmarksnumbered=true,                 %% Add numbers to bookmarks
    bookmarksopen=true,                     %% bookmarks default open
    bookmarksopenlevel=2,                   %% Level 2 bookmarks
    % backref=true,                         %% add links in bib (default)
    % pagebackref=true,                     %% in the bibliography (default)
    % hyperindex=true,                      %% in the index (default)
    breaklinks=true,                        %% break lines if long
    colorlinks=false,                       %% colour links
    urlcolor=blue,                          %% link colour
    citecolor=blue,	                        %% bibliography link color
    linkcolor=blue,	                        %% internal link color
    anchorcolor=blue,                       %% anchor link color
    linktocpage=false,                      %% pagenumber link in TOC
    pdfborder={0 0 0}                       %% pdf border settings
}